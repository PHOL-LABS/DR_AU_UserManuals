\chapter{Einführung}\label{ch:introduction}

Dieses Benutzerhandbuch beschreibt das digitale Kombiinstrument \DRAudi{} für Audi Coupe, Coupe Quattro, Quattro/Sport, 80~(B2), 90 und 4000. Es fasst die Hardwarevarianten zusammen, erläutert deren Funktionen und erklärt Einbau, Konfiguration, Betrieb, Lagerung sowie Wartung des Instruments. Die Hinweise richten sich an Fahrzeughalter, Kfz-Elektriker und Servicebetriebe, die das Produkt nachrüsten.

Die folgenden Kapitel behandeln das Modellkennzeichnungssystem, die Steckverbinderbelegung, die Betriebsbedingungen sowie die Einbauverfahren zum Ersatz der werkseitigen Instrumentierung. Referenzen zu Wartung, Konfiguration und Fehlersuche stellen sicher, dass das Kombiinstrument ohne die ursprüngliche russischsprachige Dokumentation gewartet werden kann.

\begin{figure}[htbp]
    \centering
    \begin{subfigure}{0.48\textwidth}
        \includegraphics[width=\linewidth]{figures/digifizaudi/image001.png}
        \caption{Installiertes \DRAudiGreen{}-Kombiinstrument.}
    \end{subfigure}\hfill
    \begin{subfigure}{0.48\textwidth}
        \includegraphics[width=\linewidth]{figures/digifizaudi/image005.jpg}
        \caption{Lieferumfang des GART-AG-Pakets.}
    \end{subfigure}
    \begin{subfigure}{0.48\textwidth}
        \includegraphics[width=\linewidth]{figures/digifizaudi/image003.jpg}
        \caption{Installiertes \DRAudiRed{}-Kombiinstrument.}
    \end{subfigure}\hfill
    \begin{subfigure}{0.48\textwidth}
        \includegraphics[width=\linewidth]{figures/digifizaudi/image006.jpg}
        \caption{Für den Versand vorbereiteter Paketinhalt.}
    \end{subfigure}
    \caption{Beispielhafte \DRAudi{}-Kombiinstrumente und Zubehörteile.}
\end{figure}

Jede Variante wird mit den für den jeweiligen Fahrzeugkabelbaum erforderlichen Komponenten geliefert. Die späteren Kapitel entschlüsseln die Produktkennzeichnungen, enthalten Verdrahtungstabellen und beschreiben den Konfigurationsablauf.
