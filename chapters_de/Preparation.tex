\chapter{Arbeitsvorbereitung und Arbeitsablauf}\label{ch:preparation}

\section{Vorbereitung des Fahrzeugs}
Beim Austausch des Serien-Kombiinstruments gegen das \DRAudi{}-Instrument die folgende Reihenfolge einhalten:
\begin{enumerate}
    \item Kunststoffverkleidungen im Pedalbereich und am unteren Armaturenbrett entfernen, um den Zugang zum Originalinstrument herzustellen.
    \item Fahrzeugbatterie abklemmen.
    \item Kabelbaum vom serienmäßigen Kombiinstrument abstecken.
    \item Falls vorhanden, mechanische Tachowelle lösen.
    \item Kombiinstrument aus den Haltern abschrauben und vorsichtig aus dem Fahrzeug entnehmen.
    \item Mitgelieferten Kabelsatz für den Geschwindigkeitssensor sowie optionale Temperaturfühler verlegen.
    \item \DRAudi{}-Kombiinstrument in die Führungen einsetzen und mit Schrauben befestigen.
    \item Kabelsatz des Lenksäulenschalters anschließen, wenn externe MFA-Bedienung benötigt wird.
    \item Bei alternativer Audi-Pinbelegung die Stecker gemäß \autoref{ch:description} umpinnen, bevor der Kabelbaum wieder angeschlossen wird.
    \item Steckverbinder in das Kombiinstrument einstecken.
    \item Elektronischen Geschwindigkeitssensor montieren oder die mechanische Tachowelle wieder anschließen.
    \item Armaturenbrettverkleidung und Pedalabdeckung in umgekehrter Reihenfolge montieren.
\end{enumerate}

\section{Betrieb des Kombiinstruments}
\begin{itemize}
    \item Das Kombiinstrument startet automatisch mit der Zündung. Die Standlichtschaltung steuert die Hintergrundbeleuchtung.
    \item Beim Start leuchtet zunächst die gesamte Geschwindigkeitsskala, während die interne Diagnose das Drehzahlmodell stabilisiert; danach zeigt das Display die aktuelle Leerlaufdrehzahl an.
    \item Sobald sich das Fahrzeug bewegt, werden die in \Cref{ch:technical-specs} genannten Parameter angezeigt.
\end{itemize}

\subsection{MFA-Funktionen}
Es stehen sechs MFA-Seiten zur Verfügung:
\begin{enumerate}
    \item Tagesbetriebszeit.
    \item Fahrtstrecke.
    \item Kraftstoffverbrauch (nur bei vorhandener Sensorik).
    \item Durchschnittsgeschwindigkeit (Anzeige als Wert mal zehn).
    \item Motoröltemperatur (externer Kabelsatz erforderlich).
    \item Außentemperatur (externer Kabelsatz erforderlich).
\end{enumerate}
Bei den Audi-spezifischen Varianten wird die MFA über den Lenksäulenschalter bedient; bei Fahrzeugen ohne diesen Schalter wechselt ein kurzer Druck auf die vordere Touch-Taste den Modus. Die Druckdauer wirkt wie folgt:
\begin{itemize}
    \item Kurzer Druck (\(<1\)~s): zur nächsten MFA-Funktion wechseln.
    \item Mittlerer Druck (1--3~s, wenn kein Lenksäulenschalter verbaut ist): zwischen MFA-Speicherblöcken umschalten; die Änderung wird im Display angezeigt.
    \item Langer Druck (3--7~s): aktive MFA-Funktion zurücksetzen (betrifft Verbrauch, Strecke, Zeit und Durchschnittsgeschwindigkeit).
\end{itemize}

\subsection{Hintergrundbeleuchtung und Anzeigelayout}
Das Kombiinstrument bietet eine automatische Helligkeitsregelung über eine Fotodiode sowie eine manuelle Übersteuerung über die in \Cref{ch:replica-setup,ch:replica-next-setup} beschriebenen Konfigurationsschnittstellen.

Die Anordnung der horizontalen Kontrollleuchten und die Legende sind in \autoref{fig:indicator-layout} dargestellt.

\begin{figure}[htbp]
    \centering
    \begin{subfigure}{0.48\textwidth}
        \includegraphics[width=\linewidth]{figures/digifizaudi/image017.png}
        \caption{Anzeigelayout während des Selbsttests beim Einschalten.}
    \end{subfigure}\hfill
    \begin{subfigure}{0.48\textwidth}
        \includegraphics[width=\linewidth]{figures/digifizaudi/image018.png}
        \caption{Legende der horizontalen Kontrollleuchtengruppe.}
    \end{subfigure}
    \caption{Anzeigeschema des Kombiinstruments.}
    \label{fig:indicator-layout}
\end{figure}

\subsection{Konfigurationsschnittstellen}
\begin{itemize}
    \item Bluetooth-Varianten verwenden ein Bluetooth-2.0-Modul. Installieren Sie \emph{Serial Bluetooth Terminal} aus Google Play, koppeln Sie mit dem Kombiinstrument und senden Sie Befehle direkt in der Terminalansicht. Apple-iOS-Geräte können mit diesem Modul nicht verbunden werden.
    \item miniUIOD-Varianten stellen einen integrierten WLAN-Access-Point mit Konfigurationsportal bereit, beschrieben in \Cref{ch:replica-next-setup}. Deaktivieren Sie mobile Daten während der Verbindung, damit das Captive Portal korrekt geladen wird.
\end{itemize}
Beide Generationen können zudem auf dem Prüftisch über die USBasp-Programmierschnittstelle versorgt und konfiguriert werden.
