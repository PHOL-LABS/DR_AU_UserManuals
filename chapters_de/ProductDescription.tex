\chapter{Beschreibung und Betrieb des Produkts}\label{ch:description}

\section{Zweck}
Das digitale \DRAudi{}-Kombiinstrument ersetzt das originale Audi-Instrument und erweitert dessen Funktionsumfang.
Es bietet vollständig elektronische Anzeigen für Fahrzeuggeschwindigkeit, Motordrehzahl, Kühlmittelstatus, Kraftstoffstand, MFA-Berechnungen sowie Warnleuchten.
Alle Varianten besitzen LED-basierte Displays, in jeder Konfiguration eine Drehzahlskala sowie je nach Ausführung einen Bluetooth- oder miniUIOD-WLAN-Controller für Konfiguration und Diagnose.

\section{Modellidentifikation}
Jedes Kombiinstrument ist mit dem in \autoref{tab:model-range} aufgeführten Paketcode gekennzeichnet.
Beide Varianten sind montagefertige Einheiten mit externem Kabelsatz für den elektronischen Geschwindigkeitssensor und unterscheiden sich nur in der Farbe der Frontbeleuchtung.

{\scriptsize
\begin{tblr}{
    colspec={Q[l,2.2cm] X[l]},
    hlines,
    row{1} = {font=\bfseries}
}
Modell & Beschreibung \\
GART-AR & Anwendungen für Ottomotoren, montierte Einheit mit externem elektronischem Geschwindigkeitssensor, Doppelstecker, rote Beleuchtung. \\
GART-AG & Anwendungen für Ottomotoren, montierte Einheit mit externem elektronischem Geschwindigkeitssensor, Doppelstecker, grüne Beleuchtung. \\
\end{tblr}}
\label{tab:model-range}

\section{Steckverbinderbelegung}
\subsection{Hauptstecker 10-polig}
Der 10-polige Stecker bündelt die negativen Rückleitungen der Kontrollleuchten und Sensorleitungen.
Alle positiven Versorgungen kommen vom zugehörigen 12-poligen Stecker.

{\scriptsize
\begin{tblr}{
    colspec={Q[c,1.2cm] X[l]},
    hlines,
    row{1} = {font=\bfseries}
}
Pin & Belegung \\
1 & Rückleitung Warnleuchte Kühlmittel (Temperaturanzeige negativ). \\
2 & Ausgang/Eingang Spannungsstabilisator (normalerweise unbenutzt). \\
3 & Masse (KL~31). \\
4 & Nicht belegt. \\
5 & Eingang Kühlmitteltemperatursensor. \\
6 & Rückleitung ABS-Warnleuchte (\enquote{ABS AUS}). \\
7 & Rückleitung Treiber Temperaturwarnleuchte (Blinkerplatine \enquote{L}). \\
8 & Rückleitung Öldruckwarnleuchte. \\
9 & Rückleitung Bremswarnleuchte. \\
10 & Rückleitung Batterie/Ladekontrollleuchte. \\
\end{tblr}}

\subsection{Hauptstecker 12-polig}
Der zweite Stecker führt die positiven Versorgungen und Sensoreingänge, die das Kombiinstrument benötigt.

{\scriptsize
\begin{tblr}{
    colspec={Q[c,1.2cm] X[l]},
    hlines,
    row{1} = {font=\bfseries}
}
Pin & Belegung \\
1 & Gemeinsame Plusversorgung für Kontrollleuchten (Blinker, Batterie, Handbremse, Öl, Kühlmittel) --- KL~15. \\
2 & Nicht belegt. \\
3 & Rückleitung Blinkeranzeige (links/rechts kombiniert). \\
4 & Plusversorgung Warnblinkanlage. \\
5 & Plusversorgung Kontrollleuchte Heckscheibenheizung. \\
6 & Plusversorgung Kontrollleuchte Frontlampenheizung/Nebel. \\
7 & Plusversorgung Fernlichtkontrollleuchte. \\
8 & Eingang Kraftstoffstandgeber. \\
9 & Drehzahleingang (KL~1 Zündsignal). \\
10 & Dauerplus +12~V (KL~30). \\
11 & Versorgung Blinkgeberlampe (Brücke zu Pin~1). \\
12 & Versorgung Hintergrundbeleuchtung. \\
\end{tblr}}

\section{Kabelbaum Lenksäulenschalter}
Der Zusatzkabelbaum für den Lenksäulenschalter stellt sechs Leitungen bereit:
\begin{enumerate}
    \item MFA-Modusauswahl.
    \item Auswahl MFA-Speicherblock.
    \item MFA-Reset.
    \item Außentemperatursensor (optional).
    \item Öltemperatursensor (optional).
    \item Fahrzeugmasse.
\end{enumerate}

\section{Alternative Audi-Verdrahtungsvariante}
Bei einigen Audi-Fahrzeugen wird eine abweichende Steckerbelegung verwendet.
In diesem Fall ist der Fahrzeugkabelbaum auf die oben beschriebene Standardbelegung umzupinnen.
Die werkseitige Konfiguration entspricht typischerweise den folgenden Tabellen.

\subsection{Alternative 10-polige Belegung}
{\scriptsize
\begin{tblr}{
    colspec={Q[c,1.2cm] X[l]},
    hlines,
    row{1} = {font=\bfseries}
}
Pin & Belegung \\
1 & Ausgang Spannungsstabilisator (Plus, unbenutzt). \\
2 & Nicht belegt. \\
3 & Nicht belegt. \\
4 & Gemeinsame Masse für Uhr, Hintergrundbeleuchtung und Stabilisator (KL~31). \\
5 & Fernlichtkontrollleuchte. \\
6 & Öldruckwarnleuchte. \\
7 & Blinkereingang (KL~49). \\
8 & Rückleitung Blinkeranzeige. \\
9 & Kühlmitteltemperatursensor. \\
10 & Warnleuchte Lambdasonde. \\
\end{tblr}}

\subsection{Alternative 12-polige Belegung}
{\scriptsize
\begin{tblr}{
    colspec={Q[c,1.2cm] X[l]},
    hlines,
    row{1} = {font=\bfseries}
}
Pin & Belegung \\
1 & Warnleuchte Sicherheitsgurt. \\
2 & Warnleuchte Feststellbremse. \\
3 & Warnleuchte Warnblinkanlage. \\
4 & Masse (KL~31). \\
5 & Warnleuchte Lichtmaschine. \\
6 & Kraftstoffstandgeber. \\
7 & EGR-Anzeige (+, unbenutzt). \\
8 & EGR-Anzeige (-, unbenutzt). \\
9 & Zündungsplus +12~V (KL~15). \\
10 & Drehzahleingang (KL~1). \\
11 & Versorgung Uhr (KL~31). \\
12 & Versorgung Hintergrundbeleuchtung. \\
\end{tblr}}

\subsection{Umpinnverfahren}
Führen Sie die Umrüstung Schritt für Schritt aus:
\begin{enumerate}
    \item Kontakte mit kleinem Schraubendreher oder Entriegelungswerkzeug aus den Fahrzeugsteckern lösen.
    \item Prüfen, dass Pin~4 an 10-poligem und 12-poligem Stecker mit Karosseriemasse verbunden ist.
    \item Gewählte Masseleitung von Pin~4 auf Pin~3 des 10-poligen Steckers umsetzen.
    \item Leitung von Pin~9 des 12-poligen Steckers auf Pin~1 desselben Steckers verlegen.
    \item Beide Fahrzeugstecker am Kombiinstrument anschließen, Zündung einschalten und vor dem Fortfahren die Spannungsversorgung des Panels prüfen.
    \item Pin~1 des 10-poligen Steckers mit Pin~2 desselben Steckers brücken.
    \item Pin~5 des 10-poligen Steckers mit Pin~7 des 12-poligen Steckers verbinden (Fernlicht) und Anzeige prüfen.
    \item Pin~6 des 10-poligen Steckers auf Pin~8 des 10-poligen Steckers führen.
    \item Pin~8 des 10-poligen Steckers auf Pin~3 des 12-poligen Steckers umsetzen; Blinkerfunktion prüfen. Falls inaktiv, stattdessen Pin~7 verwenden.
    \item Pin~9 des 10-poligen Steckers mit Pin~5 desselben Steckers verbinden (Kühlmitteltemperatursensor).
    \item Warnleuchte Lambdasonde (10-polig Pin~10) getrennt lassen, sofern kein optionaler interner Stecker genutzt wird.
    \item Warnleuchten für Gurt und Handbremse (12-polig Pins~1 und~2) sind am internen Zusatzstecker verfügbar; Anschluss optional.
    \item Pin~3 des 12-poligen Steckers mit Pin~4 desselben Steckers brücken.
    \item Pin~5 des 12-poligen Steckers mit Pin~10 des 10-poligen Steckers verbinden (Lichtmaschinenlampe).
    \item Pin~6 des 12-poligen Steckers mit Pin~8 des 12-poligen Steckers verbinden (Kraftstoffgeber).
    \item Pin~10 des 12-poligen Steckers mit Pin~9 desselben Steckers brücken.
    \item Pin~12 des 12-poligen Steckers unverändert belassen.
    \item Nicht verwendete Leitungen kennzeichnen und sicher befestigen.
\end{enumerate}

\begin{figure}[htbp]
    \centering
    \includegraphics[width=0.75\textwidth]{figures/digifizaudi/image009.png}
    \caption{Umpinndiagramm für Fahrzeuge mit alternativer Audi-Kabelbaumbelegung.}
\end{figure}

\section{Service-Steckverbinder}
Ein 14-poliger Service-Steckverbinder im Gehäuse bildet die externe Verdrahtung ab und stellt zusätzliche Diagnosepunkte bereit.
Die Pinbelegung unterscheidet sich geringfügig zwischen roter und grüner Variante; beide sind unten zusammengefasst.

{\scriptsize
\begin{tblr}{
    colspec={Q[c,0.8cm] X[l]},
    hlines,
    row{1} = {font=\bfseries},
}
\textbf{Pin} & \textbf{\DRAudiRed{} Service-Steckverbinder} \\
1 & Sicherheitsgurtanzeige (+). \\
2 & Eingang Geschwindigkeitssensor. \\
3 & Fahrzeugmasse. \\
4 & Anzeige Heckscheibenheizung (+). \\
5 & Eingang linker Blinker (BLL). \\
6 & Eingang rechter Blinker (BLR). \\
7 & Zündungsversorgung. \\
8 & Bremsanzeige (+). \\
9 & Anzeige Frontlampenheizung (+). \\
10 & Reservierte Fehleranzeige (+). \\
11 & ABS-Anzeige (+). \\
12 & Reserviert. \\
13 & Check-Engine-Anzeige (-). \\
14 & Reserviert. \\
\end{tblr}}

{\scriptsize
\begin{tblr}{
    colspec={Q[c,0.8cm] X[l]},
    hlines,
    row{1} = {font=\bfseries},
}
\textbf{Pin} & \textbf{\DRAudiGreen{} Service-Steckverbinder} \\
1 & Nicht belegt. \\
2 & Eingang Geschwindigkeitssensor. \\
3 & Fahrzeugmasse. \\
4 & Anzeige Heckscheibenheizung (+). \\
5 & Eingang linker Blinker (BLL). \\
6 & Eingang rechter Blinker (BLR). \\
7 & Zündungsversorgung. \\
8 & Bremsanzeige (+). \\
9 & Anzeige Frontlampenheizung (+). \\
10 & Reservierte Fehleranzeige (+). \\
11 & ABS-Anzeige (+). \\
12 & Reserviert. \\
13 & Check-Engine-Anzeige (-). \\
14 & Reserviert. \\
\end{tblr}}

\section{Mitgelieferte Ausstattung}
Jeder Liefersatz enthält die Kombiinstrumenteinheit, einen USBasp-Programmer und einen Kabelsatz für den Geschwindigkeitssensor.
Elektronische Geschwindigkeitssensoren oder Adapter für den Wellenantrieb werden entsprechend dem gewählten Fahrzeugkit mitgeliefert.
