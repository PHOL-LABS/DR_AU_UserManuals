\chapter{miniUIOD-Konfiguration und Wartung}\label{ch:replica-next-setup}

Das miniUIOD-Modul ergänzt das \DRAudi{}-Kombiinstrument um eine WLAN-Konfigurationsschnittstelle.
Die Hardwareanordnung ist in \autoref{fig:next-hardware} dargestellt.

\begin{figure}[htbp]
    \centering
    \includegraphics[width=0.6\textwidth]{figures/digifizaudi/image019.png}
    \caption{Im \DRAudi{}-Kombiinstrument eingebauter miniUIOD-Controller.}
    \label{fig:next-hardware}
\end{figure}

\section{Handhabung des Panels}
\begin{itemize}
    \item Die UV-bedruckte Polycarbonatfront vor Kratzern und Fremdkörpern schützen. Größere Schäden erfordern Ersatzteile von PHOL-LABS Kft und gelten nicht als Gewährleistungsfall.
    \item Die Echtzeituhr wird über das WLAN-Bedienfeld konfiguriert. Sie wird zurückgesetzt, sobald die Dauerstromversorgung unterbrochen ist.
\end{itemize}

\section{WLAN-Steuerportal}
Konfiguration, Datenerfassung und Firmwareverwaltung erfolgen über die integrierte Webanwendung.
\begin{itemize}
    \item Mit dem WLAN-Access-Point des Kombiinstruments verbinden. Mobile Daten deaktivieren und \texttt{Digifiz\_AP} (Passwort \texttt{87654321}) beitreten; einige Revisionen senden \texttt{PHOL-LABS2} mit demselben Passwort.
    \item Standard-IP ist \texttt{192.168.4.1}. Wenn das Gerät in ein anderes Netzwerk eingebunden wurde, im Subnetz mit einem IP-Tool nach einer Adresse mit Endung \texttt{.32} suchen.
    \item Das Portal enthält die Register \emph{WiFi}, \emph{Control}, \emph{Settings}, \emph{Colors} und \emph{About} (\autoref{fig:next-control-tabs}). Im Register \emph{WiFi} werden Netzwerkeinstellungen und Firmware-Uploads verwaltet; \emph{Control} dient zur Parameteränderung; \emph{Settings} bietet einen strukturierten Editor für Firmwareparameter; \emph{Colors} verwaltet segmentierte Farbschemata; \emph{About} enthält Autoreninformationen.
\end{itemize}

\begin{figure}[htbp]
    \centering
    \begin{subfigure}{0.48\textwidth}
        \includegraphics[width=\linewidth]{figures/digifizaudi/image020.png}
        \caption{Übersicht Register \emph{Control}.}
    \end{subfigure}\hfill
    \begin{subfigure}{0.48\textwidth}
        \includegraphics[width=\linewidth]{figures/digifizaudi/image021.png}
        \caption{Nummerierte Bedienelemente und Befehlseingaben.}
    \end{subfigure}
    \caption{miniUIOD-WLAN-Bedienoberfläche.}
    \label{fig:next-control-tabs}
\end{figure}

\section{Befehlseingabe}
Im Register \emph{Control} befinden sich Befehlszeile (1), \emph{Process}-Schaltfläche (2), Ergebnisfenster (3), Schnellbedienung (4), \emph{Save}-Schaltfläche (5) und \emph{Reset}-Schaltfläche (6).
Befehle als Ganzzahlpaare \verb|<number> <value>| eingeben; Satzzeichen und Anführungszeichen sind nicht erforderlich.
\autoref{fig:next-command-example} zeigt das Umschalten der automatischen Helligkeit.

\begin{figure}[htbp]
    \centering
    \includegraphics[width=0.55\textwidth]{figures/digifizaudi/image022.png}
    \caption{Beispielsequenz zum Deaktivieren der automatischen Helligkeit.}
    \label{fig:next-command-example}
\end{figure}

\section{Befehlsreferenz}
\begin{table}[htbp]
    \centering
    \caption{Wichtige miniUIOD-Konfigurationsbefehle.}
    \label{tbl:next-commands}
    {\scriptsize
    \begin{tblr}{
        colspec = {Q[c,0.14\linewidth] Q[l,0.34\linewidth] Q[l,0.5\linewidth]},
        rowsep = 2pt,
    }
        \toprule
        \textbf{Befehl} & \textbf{Name} & \textbf{Beschreibung} \\
        \midrule
        22 (oder 0) & \paramname{PARAMETER\_RPMCOEFFICIENT} & Kalibrierfaktor Motordrehzahl (100--10000). \\
        1  & \paramname{PARAMETER\_SPEEDCOEEFICIENT} & Kalibrierfaktor Geschwindigkeit (10--255). \\
        2  & \paramname{PARAMETER\_COOLANTTHERMISTORB} & Beta-Koeffizient Kühlmittelthermistor (2000--5000). \\
        3  & \paramname{PARAMETER\_OILTHERMISTORB} & Beta-Koeffizient Ölthermistor (2000--5000). \\
        4  & \paramname{PARAMETER\_AIRTHERMISTORB} & Beta-Koeffizient Außenthermistor (2000--5000). \\
        5  & \paramname{PARAMETER\_TANKMINRESISTANCE} & Minimaler Widerstand Kraftstoffgeber (0--1000~\ohm). \\
        6  & \paramname{PARAMETER\_TANKMAXRESISTANCE} & Maximaler Widerstand Kraftstoffgeber (100--1000~\ohm). \\
        7  & \paramname{PARAMETER\_TAU\_COOLANT} & Filterkonstante Kühlmitteltemperatur (1--50). \\
        8  & \paramname{PARAMETER\_TAU\_OIL} & Filterkonstante Öltemperatur (1--50). \\
        9  & \paramname{PARAMETER\_TAU\_AIR} & Filterkonstante Außentemperatur (1--50). \\
        10 & \paramname{PARAMETER\_TAU\_TANK} & Filterkonstante Kraftstoffstand (1--50). \\
        11 & \paramname{PARAMETER\_MILEAGE} & Gesamtkilometerstand (0--999999). \\
        12 & \paramname{PARAMETER\_DAILY\_MILEAGE} & Tageskilometerzähler (0--9999). \\
        13 & \paramname{PARAMETER\_AUTO\_BRIGHTNESS} & Automatische Helligkeit (1=ein, 0=aus). \\
        14 & \paramname{PARAMETER\_BRIGHTNESS\_LEVEL} & Manuelle Helligkeit (0--60\%; Werte über 60 verkürzen LED-Lebensdauer). \\
        15 & \paramname{PARAMETER\_TANK\_CAPACITY} & Tankkapazität in Litern (0--99). \\
        16 & \paramname{PARAMETER\_MFA\_STATE} & Aktiver MFA-Modus (normal über Hardwareeingang gesteuert). \\
        17 & \paramname{PARAMETER\_BUZZER\_OFF} & Summer deaktivieren (1 aus, 0 ein). \\
        18 & \paramname{PARAMETER\_MAX\_RPM} & Drehzahlmesserskalierung (typisch 7000 oder 8000). \\
        19 & \paramname{PARAMETER\_NORMAL\_RESISTANCE\_COOLANT} & Widerstand Kühlmittelsensor bei \SI{25}{\celsius} (1000--10000~\ohm). \\
        20 & \paramname{PARAMETER\_NORMAL\_RESISTANCE\_OIL} & Widerstand Ölsensor bei \SI{25}{\celsius} (1000--10000~\ohm). \\
        21 & \paramname{PARAMETER\_NORMAL\_RESISTANCE\_AMB} & Widerstand Außensensor bei \SI{25}{\celsius} (1000--10000~\ohm). \\
        23 & \paramname{PARAMETER\_DOT\_OFF} & Verhalten Doppelpunkt der Uhr (0=blinkt, 1=dauerhaft). \\
        24 & \paramname{PARAMETER\_BACKLIGHT\_ON} & Hintergrundbeleuchtung mit Abblendlicht aktivieren. \\
        25 & \paramname{PARAMETER\_M\_D\_FILTER} & Medianfilterkonstante (legacy, meist ungenutzt). \\
        26 & \paramname{PARAMETER\_COOLANT\_MAX\_R} & Kühlmittelschwelle für Vollanschlag. \\
        27 & \paramname{PARAMETER\_COOLANT\_MIN\_R} & Kühlmittelschwelle für \enquote{1~bar}. \\
        31--33 & \paramname{PARAMETER\_MAINCOLOR\_[RGB]} & Farbkomponenten der Benutzeroberfläche. \\
        37 & \paramname{PARAMETER\_RPM\_FILTER} & Filterstärke Drehzahl (10--200). \\
        128 & \paramname{PARAMETER\_READ\_ADDITION} & 128 addieren, um aktuellen Wert auszulesen. \\
        255 & \paramname{PARAMETER\_SET\_HOUR} & Uhrstunden setzen (24-h-Format). \\
        254 & \paramname{PARAMETER\_SET\_MINUTE} & Uhrminuten setzen. \\
        253 & \paramname{PARAMETER\_RESET\_DAILY\_MILEAGE} & Tageskilometerzähler zurücksetzen. \\
        252 & \paramname{PARAMETER\_RESET\_DIGIFIZ} & Werksreset gespeicherter Parameter. \\
        \bottomrule
    \end{tblr}}
\end{table}

\section{Standardwerte}
\begin{table}[htbp]
    \centering
    \caption{Standard-miniUIOD-Einstellungen für das \DRAudi{}-Kombiinstrument.}
    \label{tbl:next-defaults}
    {\scriptsize
    \begin{tblr}{
        colspec = {Q[l,0.42\linewidth] Q[c,0.15\linewidth] Q[l,0.43\linewidth]},
        rowsep = 2pt,
    }
        \toprule
        \textbf{Parameter} & \textbf{Standard} & \textbf{Hinweise} \\
        \midrule
        \paramname{PARAMETER\_RPMCOEFFICIENT} & 3000 & Typisch für Audi-Zündsignale. \\
        \paramname{PARAMETER\_SPEEDCOEEFICIENT} & 100 & Kalibriert für \SI{100}{\kilo\metre\per\hour}. \\
        \paramname{PARAMETER\_COOLANTTHERMISTORB} & 4000 &  \\
        \paramname{PARAMETER\_OILTHERMISTORB} & 4000 &  \\
        \paramname{PARAMETER\_AIRTHERMISTORB} & 3812 &  \\
        \paramname{PARAMETER\_TANKMINRESISTANCE} & 35 & \si{\ohm}. \\
        \paramname{PARAMETER\_TANKMAXRESISTANCE} & 265 & \si{\ohm}. \\
        \paramname{PARAMETER\_TAU\_COOLANT} & 2 & Filterkonstante. \\
        \paramname{PARAMETER\_TAU\_OIL} & 2 & Filterkonstante. \\
        \paramname{PARAMETER\_TAU\_AIR} & 2 & Filterkonstante. \\
        \paramname{PARAMETER\_TAU\_TANK} & 2 & Filterkonstante. \\
        \paramname{PARAMETER\_MILEAGE} & Fahrzeugspezifisch & Gespeicherter Kilometerstand bleibt erhalten. \\
        \paramname{PARAMETER\_DAILY\_MILEAGE} & 0 &  \\
        \paramname{PARAMETER\_AUTO\_BRIGHTNESS} & 1 & Aktiv. \\
        \paramname{PARAMETER\_BRIGHTNESS\_LEVEL} & 25 & Manuelle Helligkeit (bei deaktivierter Automatik anpassen). \\
        \paramname{PARAMETER\_TANK\_CAPACITY} & 63 & Liter. \\
        \paramname{PARAMETER\_MFA\_STATE} & 0 & Standard-MFA-Seite. \\
        \paramname{PARAMETER\_BUZZER\_OFF} & 1 & Summer aus. \\
        \paramname{PARAMETER\_MAX\_RPM} & 8000 & Drehzahlmesserskala. \\
        \paramname{PARAMETER\_NORMAL\_RESISTANCE\_COOLANT} & 1000 & \si{\ohm} bei \SI{25}{\celsius}. \\
        \paramname{PARAMETER\_NORMAL\_RESISTANCE\_OIL} & 1000 & \si{\ohm} bei \SI{25}{\celsius}. \\
        \paramname{PARAMETER\_NORMAL\_RESISTANCE\_AMB} & 2991 & \si{\ohm} bei \SI{25}{\celsius}. \\
        \paramname{PARAMETER\_DOT\_OFF} & 0 & Blinkender Uhren-Doppelpunkt. \\
        \paramname{PARAMETER\_BACKLIGHT\_ON} & 1 & Hintergrundbeleuchtung mit Abblendlicht aktiv. \\
        \paramname{PARAMETER\_M\_D\_FILTER} & 65535 & Legacy-Medianfilterkonstante. \\
        \paramname{PARAMETER\_COOLANT\_MAX\_R} & 120 & \si{\celsius}. \\
        \paramname{PARAMETER\_COOLANT\_MIN\_R} & 60 & \si{\celsius}. \\
        \paramname{PARAMETER\_MAINCOLOR\_R} & 180 & Gelb-grüner Standardton. \\
        \paramname{PARAMETER\_MAINCOLOR\_G} & 240 & Gelb-grüner Standardton. \\
        \paramname{PARAMETER\_MAINCOLOR\_B} & 6 & Gelb-grüner Standardton. \\
        \paramname{PARAMETER\_RPM\_FILTER} & 70 & Filterverhalten. \\
        \paramname{PARAMETER\_UPTIME} & 0 & Laufzeitzähler. \\
        \bottomrule
    \end{tblr}}
\end{table}

\section{Parameter auslesen und Beispiele}
Zum Auslesen eines Parameters 128 zur Befehlsnummer addieren (z.~B. \verb|129 0| für den Geschwindigkeitskoeffizienten).
Typische Befehle: Automatikhelligkeit aus (\verb|13 0|), wieder ein (\verb|13 1|), Geschwindigkeitskoeffizient anpassen (\verb|1 110| erhöht die Anzeige um 10\%), Kilometerstand setzen (\verb|11 123456|).
Uhrzeit mit \verb|255 <hours>| und anschließend \verb|254 <minutes>| einstellen.

\section{Servicebefehle}
Neuere Firmware akzeptiert lesbare Parameternamen, z.~B. \verb|PARAMETER_RPMCOEFFICIENT 3000|.
Der Diagnosebefehl \verb|adc 0| gibt Roh-ADC-Werte für die Sensoranalyse aus.
Firmwareaktualisierungen ergänzen visuelle Farbsteuerungen; daher regelmäßig über das Register \emph{WiFi} aktualisieren.

\section{Parametereditor im Register Settings}
Das Register \emph{Settings} spiegelt die Parameterliste aus \autoref{tbl:next-commands} und \autoref{tbl:next-defaults} und ergänzt Bereiche, Beschreibungen sowie Datentypen.
Nutzen Sie es, wenn Sie lieber grafisch statt über Befehlsnummern arbeiten.

\section{Benutzerdefinierte Farbschemata}
Im Register \emph{Colors} steht ein segmentbasierter Editor für die WS2812-LED-Hintergrundbeleuchtung zur Verfügung.
Mit \emph{Load Scheme} aktive Zuordnung laden, Bereiche anpassen und mit \emph{Set Scheme} hochladen.
Direkt danach in \emph{Control} wechseln und \emph{Save parameters} drücken, sonst geht das Schema nach Neustart verloren.
Optional erlauben Export-/Import-Schaltflächen eine JSON-Sicherung des Layouts.

Wenn die benutzerdefinierte Farbschema-Option später im Register \emph{Settings} deaktiviert wird, fällt das Kombiinstrument auf den klassischen Einfarbmodus mit \verb|PARAMETER_MAINCOLOR_R/G/B| zurück.
