\chapter{Betriebsbedingungen und Sicherheitshinweise}\label{ch:safety}

\section{Umgebungsgrenzen}
\begin{itemize}
    \item Das Kombiinstrument arbeitet zwischen \(-40\,^{\circ}\mathrm{C}\) und \(+70\,^{\circ}\mathrm{C}\) bei relativer Luftfeuchte bis 95~\%.
    \item Das Gerät kann ganzjährig im Fahrzeug eingebaut bleiben, auch bei längeren Standzeiten.
\end{itemize}

\section{Sicherheitsmaßnahmen}
\begin{enumerate}
    \item Das \DRAudi{}-Kombiinstrument ist ein Do-it-yourself-Produkt, das von Enthusiasten montiert und integriert wird. Beachten Sie bei allen Arbeiten die allgemeinen Regeln der elektrischen Sicherheit.
    \item Das Produkt ist für persönliche Projekte von Fahrzeughaltern vorgesehen.
    \item Die Anzeigen sind nicht zertifiziert bzw. eichrechtlich geprüft, entsprechen jedoch den bei Veröffentlichung angegebenen Spezifikationen.
    \item Verwenden Sie das Kombiinstrument nur, wenn Sie die Verantwortung für Einbau und Verkehrssicherheit übernehmen.
    \item Wenn Anzeigewerte nicht plausibel sind, vergleichen Sie mit den Serieninstrumenten oder externen Messgeräten.
    \item Verwenden Sie die Ausgänge des Kombiinstruments nicht zur automatischen Fahrzeugsteuerung.
    \item Die Autoren übernehmen keine Haftung für Folgen aus Einbau oder Nutzung, einschließlich Bußgeldern oder Unfällen. Innerhalb der Gewährleistungsfrist gemeldete Mängel (ein Jahr bei gemeinsam mit den Autoren ausgeführter Installation, zwei Wochen bei eigenständiger Installation) werden behoben.
    \item Die in \Cref{ch:technical-specs} angegebenen Funktionen werden für ein Jahr bei begleiteter Installation und für zwei Wochen bei eigenständiger Installation gewährleistet.
\end{enumerate}
