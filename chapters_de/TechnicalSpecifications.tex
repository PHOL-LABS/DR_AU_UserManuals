\chapter{Technische Daten}\label{ch:technical-specs}

Das \DRAudi{}-Kombiinstrument hat im ausgeschalteten Zustand keinen Leckstrom und arbeitet im Bordnetzbereich von 9~V bis 24~V~DC.

\section{Messfunktionen}
\begin{itemize}
    \item \textbf{Fahrzeuggeschwindigkeit:} Messung über die werkseitige Tachowelle oder den elektronischen Geschwindigkeitssensor. Der systematische Fehler beträgt 10~km/h, der relative Fehler 3~km/h; die Anzeige sättigt bei 999~km/h (bzw. mph bei imperialen Einheiten).
    \item \textbf{Motordrehzahl:} Ermittlung aus dem Zündsignal über eine Optokopplerstufe mit 430~nF/1.2~k\ensuremath{\Omega}-RC-Netzwerk und Diodenbegrenzung. Absoluter und relativer Fehler liegen innerhalb von 200~rpm.
    \item \textbf{Kraftstoffstand:} Erfassung über den resistiven Tanksensor mit einer Unsicherheit von etwa 10~Litern.
    \item \textbf{Kühlmitteltemperatur:} qualitative Anzeige über den serienmäßigen Thermistor im Fahrzeugkabelbaum; kein quantitativer Zahlenwert.
    \item \textbf{Uhrzeit:} Ganggenauigkeit innerhalb einer Minute.
    \item \textbf{Kontrollleuchten:} Blinker, Fernlicht, Öldruckwarnung, Generatorstatus, Handbremse, Heckscheibenheizung oder Diesel-Glühkerze sowie vordere und hintere Nebelscheinwerfer.
\end{itemize}
