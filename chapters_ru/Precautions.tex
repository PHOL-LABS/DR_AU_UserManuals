\chapter{Precautions}\label{ch:precautions}

This chapter consolidates the mandatory safety rules that accompany every \DRAudi{} instrument cluster. Ignoring any of these items is the fastest way to damage the electronics or obtain unreliable readings.

\begin{enumerate}
    \item \textbf{Disconnect the vehicle battery before starting the installation.} Working on a powered harness feels faster, but several dashboards have already been destroyed by short circuits caused by a live loom.
    \item \textbf{Never feed the sensor inputs with an external voltage source.} The coolant temperature, oil temperature, outside temperature, and fuel level channels are designed for passive sensors only. Even a “harmless” test through a resistor burns the measurement circuitry.
    \item \textbf{Remember that the dashboard contains no internal fuse.} The first protective element is the vehicle fuse box and it reacts too late to save the cluster from wiring mistakes.
    \item \textbf{Shield the unit from direct sunlight.} Prolonged exposure washes out the LCD segments and permanently reduces contrast.
    \item \textbf{Do not attempt to overdrive the LED backlight.} The illumination uses fixed-current drivers. If the daytime image is dim, add shading around the binnacle rather than increasing the drive current.
    \item \textbf{Beware of resonances in cable-driven speedometers.} Mechanical drives often oscillate at 40--60~km/h. Fit the supplied electronic sensor whenever possible.
    \item \textbf{Plan external MFA controls where required.} Vehicles without the Audi steering-column switch need the auxiliary harness connected so MFA functions remain accessible.
\end{enumerate}

\medskip
