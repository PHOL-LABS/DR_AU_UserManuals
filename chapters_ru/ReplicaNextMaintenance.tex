\chapter{Настройка и обслуживание miniUIOD}\label{ch:replica-next-setup}

Модуль miniUIOD предоставляет web-интерфейс настройки панели \DRAudi{} по Wi-Fi.

\section{Подключение к интерфейсу}
\begin{enumerate}
    \item Включите зажигание и дождитесь запуска точки доступа \texttt{Digifiz\_AP} (или \texttt{PHOL-LABS2}).
    \item Подключитесь к сети со смартфона/ноутбука.
    \item Откройте в браузере \texttt{192.168.4.1}.
\end{enumerate}

\section{Основные вкладки настройки}
\begin{itemize}
    \item \textbf{Main/Status}: текущие параметры и состояние.
    \item \textbf{Sensors}: коэффициенты и калибровка датчиков.
    \item \textbf{Display}: яркость, режимы индикации, дополнительные параметры.
    \item \textbf{WiFi/Firmware}: сетевые настройки и обновление прошивки.
\end{itemize}

\section{Обновление прошивки}
На вкладке \emph{WiFi} выберите файл \texttt{Digifiz.bin} и запустите обновление.
Не отключайте питание до завершения процесса и автоматической перезагрузки.

\section{Обслуживание}
\begin{itemize}
    \item Периодически проверяйте стабильность соединения и версию прошивки.
    \item После обновлений выполняйте контроль показаний на холостом ходу и в движении.
\end{itemize}
